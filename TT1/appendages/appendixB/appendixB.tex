\chapter{Códigos}\label{app2:Estado-del-arte-codes}

\section{Código en Matlab para graficar la serie trigonométrica}\label{app2:trig-code-matlab}
\begin{longlisting}
	\begin{minted}[linenos, breaklines]{matlab}
		% Número de términos en la serie de Fourier
		N = 10; % Puedes modificar N para mejorar la aproximación
		
		% Vector de tiempo de -pi a pi
		t = linspace(-pi, pi, 1000);
		
		% Función original x(t) = t
		x_original = t;
		
		% Inicialización de la aproximación de la serie de Fourier
		x_aprox = zeros(size(t));
		
		% Cálculo de la serie de Fourier en forma trigonométrica
		for n = 1:N
		bn = (2 * (-1)^(n+1)) / n; % Coeficientes de seno
		x_aprox = x_aprox + bn * sin(n * t);
		end
		
		% Gráfica de la función original y su aproximación
		figure;
		plot(t, x_original, 'k', 'LineWidth', 1.5); % Función original en negro
		hold on;
		plot(t, x_aprox, 'b', 'LineWidth', 1.5); % Aproximación en azul
		legend('Función original x(t)', ['Aproximación (Forma Trigonométrica) con N = ', num2str(N), ' términos']);
		xlabel('t');
		ylabel('x(t)');
		title('Serie de Fourier (Forma Trigonométrica) de x(t) = t');
		grid on;
		hold off;
		
	\end{minted}
	\caption{Código en matlab para calcular \ref{app1:trig-coeff}}
\end{longlisting}

\section{Código en Matlab para graficar la serie compleja}\label{app2:complex-code-matlab}
\begin{longlisting}
	\begin{minted}[linenos, breaklines]{matlab}
		% Número de términos positivos y negativos en la serie de Fourier
		N = 10; % Puedes modificar N para mejorar la aproximación
		
		% Vector de tiempo de -pi a pi
		t = linspace(-pi, pi, 1000);
		
		% Función original x(t) = t
		x_original = t;
		
		% Inicialización de la aproximación de la serie de Fourier
		x_aprox = zeros(size(t));
		
		% Cálculo de la serie de Fourier en forma exponencial compleja
		for n = -N:N
		if n == 0
		continue; % Saltar n = 0 para evitar división por cero
		end
		cn = (1i / n) * (-1)^n;
		x_aprox = x_aprox + cn * exp(1i * n * t);
		end
		
		% Tomar la parte real de la aproximación (la función original es real)
		x_aprox_real = real(x_aprox);
		
		% Gráfica de la función original y su aproximación
		figure;
		plot(t, x_original, 'k', 'LineWidth', 1.5); % Función original en negro
		hold on;
		plot(t, x_aprox_real, 'b', 'LineWidth', 1.5); % Aproximación en azul
		legend('Función original x(t)', ['Aproximación (Forma Exponencial) con N = ', num2str(N), ' términos']);
		xlabel('t');
		ylabel('x(t)');
		title('Serie de Fourier (Forma Exponencial Compleja) de x(t) = t');
		grid on;
		hold off;
		
	\end{minted}
	\caption{Ejemplo de código JavaScript.}
\end{longlisting}