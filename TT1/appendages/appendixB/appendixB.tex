%
%\chapter{Simulator}\label{app:Simulator}
%
%\begin{lstlisting}[style=Matlab-editor]
%	% Numero de terminos en la serie de Fourier
%	N = 10; % Puedes modificar N para mejorar la aproximacion
%	
%	% Vector de tiempo de -pi a pi
%	t = linspace(-pi, pi, 1000);
%	
%	% Funcion original x(t) = t
%	x_original = t;
%	
%	% Inicializacion de la aproximacion de la serie de Fourier
%	x_aprox = zeros(size(t));
%	
%	% Calculo de la serie de Fourier
%	for n = 1:N
%	bn = (2 / n) * (-1)^(n + 1);
%	x_aprox = x_aprox + bn * sin(n * t);
%	end
%	
%	% Grafica de la funcion original y su aproximacion
%	figure;
%	plot(t, x_original, 'k', 'LineWidth', 1.5); % Funcion original en negro
%	hold on;
%	plot(t, x_aprox, 'r', 'LineWidth', 1.5); % Aproximacion en rojo
%	legend('Funcion original x(t)', ['Aproximacion con N = ', num2str(N), ' terminos']);
%	xlabel('t');
%	ylabel('x(t)');
%	title('Serie de Fourier de x(t) = t');
%	grid on;
%	hold off;
%	
%\end{lstlisting}
%	
%\end{document}