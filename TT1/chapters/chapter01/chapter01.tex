\chapter{Introducción}\label{ch:Introducción}
En este capítulo, examinaremos el problema que se abordará y los objetivos previstos para este proyecto. Además, investigaremos los productos existentes que desempeñan una función similar al proyecto que se llevará a cabo.

\section{Antecedentes}
Aquí voy a buscar los antecedentes

\section{Planteamiento del Problema}
Las series de Fourier, desde su concepción por Jean-Baptiste Joseph Fourier a principios del siglo XIX, han desempeñado un rol crucial en el análisis y la comprensión de señales y fenómenos periódicos. Las series de Fourier son parte esencial en campos como la ingeniería eléctrica, la física teórica y el procesamiento de señales, imágenes y audio, permitiendo descomponer funciones periódicas en sumas de senos y cosenos, lo que facilita su análisis y manipulación. Estas herramientas matemáticas no solo han impulsado avances significativos en la ciencia y tecnología, transformando nuestra interacción con el mundo, sino que también son cruciales en la enseñanza de los fundamentos teóricos de la ingeniería. En el ámbito de las ciencias computacionales, el estudio del análisis de Fourier se aplica a áreas vitales como la adquisición y procesamiento de señales, el procesamiento de imágenes y la robótica. Esta capacidad de simplificar señales complejas en componentes básicos no solo mejora el análisis, sino que también facilita la síntesis de nuevas tecnologías que se adaptan a necesidades y entornos cambiantes~\cite{almira2017fourier}.

Este amplio espectro de aplicaciones destaca la importancia crítica de las series de Fourier no solo en la investigación avanzada, sino también en la formación académica de futuros ingenieros y científicos. Sin embargo, a pesar de su prevalencia en el currículo educativo, la implementación práctica de este análisis en entornos de aprendizaje a menudo revela áreas de oportunidad dentro de las herramientas disponibles, lo que impacta directamente en la eficacia con la que los estudiantes pueden aplicar y profundizar su comprensión de estos conceptos esenciales.

La necesidad de una herramienta más eficiente se hizo evidente durante los cursos de Matemáticas Avanzadas para la Ingeniería y Procesamiento de Señales Digitales. Al enfrentar el análisis de Fourier, especialmente al resolver ejercicios sobre series de Fourier, se descubrió la falta de medios óptimos para verificar los resultados de manera directa y eficiente. A diferencia de otros cálculos matemáticos, donde las calculadoras especializadas permiten comprobaciones rápidas y fiables, las series de Fourier requerían un proceso mucho más tedioso. Para asegurar la corrección de los cálculos, era necesario utilizar software de resolución matemática para obtener los coeficientes y, posteriormente, otro software de graficación para visualizar si los resultados correspondían efectivamente a la función dada. Este desafío se ampliaba aún más cuando se presentaban variaciones mínimas en los ejercicios, como cambiar el intervalo de la función o alternar entre series trigonométricas y complejas, o incluso al aplicar extensiones pares o impares. Cada una de estas variaciones obligaba a repetir todos los pasos desde el inicio, lo que no solo consumía tiempo valioso, sino que también complicaba la gestión del tiempo disponible, especialmente cuando se tiene una carga académica intensa. Este tiempo podría utilizarse mejor en comprender los conceptos subyacentes y explorar en profundidad el por qué y el cómo de los fenómenos analizados mediante estas series.

\section{Justificación}
Aquí puedo poner la justificación, probablemente la del Protocolo

\section{Solución Propuesta}
Aquí puedo poner la justificación, probablemente la del Protocolo


\section{Objetivos}
\subsection{Objetivo General}
Desarrollar un prototipo de una aplicación web capaz de calcular las series de Fourier en su forma trigonométrica o exponencial compleja de una función continua en un intervalo o definida a trozos, así como ser capaz de obtener la extensión par o impar de dicha función y, finalmente, graficar tanto la función original como la función expandida como una serie de Fourier y poder añadir o eliminar términos de dicha forma para apreciar como esta se aproxima a la función original.

\subsection{Objetivos Específicos}
\begin{itemize}
	\item Implementar una interfaz de usuario que permita ingresar la función continua o a trozos, seleccionar intervalos y definir el tipo de serie de Fourier (trigonométrica o exponencial) o tipo de expansión (par o impar).
	\item Implementar los módulos para el cálculo de coeficientes de la serie trigonométricas y serie exponencial compleja, adaptándose a funciones continuas o definidas a trozos.
	\item  Desarrollar un módulo de visualización que permita graficar tanto la función original como la
	aproximación de la serie de Fourier. Este módulo debería incluir opciones para añadir o eliminar términos y visualizar cómo estas modificaciones afectan la aproximación a la función original
\end{itemize}

\section{Metodología de la Investigación}
Acá voy a detallar el porqué voy a usar la metodología de Prototipos

\section{Estructura del documento técnico}
Al final voy a detallar como es que está estructurado todo el documento del TT

