\chapter{Introducción}\label{ch:Introducción}
En este capítulo, examinaremos el problema que se abordará, la necesidad e impulso para hacerlo, así como los objetivos previstos para este proyecto y nuestras propuestas de solución para esta problemática. 

\section{Motivación}
Las matemáticas son fundamentales en la formación de un ingeniero, proporcionando herramientas esenciales para modelar y resolver problemas complejos. Durante mi formación en la Escuela Superior de Cómputo, al estudiar el análisis de Fourier y resolver series de Fourier, noté la ausencia de métodos eficientes para verificar los resultados de manera directa. A diferencia de otros cálculos matemáticos que cuentan con calculadoras especializadas, las series de Fourier requieren procesos un tanto más elaborados que implican el uso de múltiples softwares para calcular coeficientes y graficar funciones, consumiendo tiempo valioso que podría dedicarse a una comprensión más profunda de los conceptos. Esta situación motiva el desarrollo de herramientas más óptimas para facilitar y agilizar la verificación de series de Fourier, mejorando así la eficiencia en el aprendizaje y aplicación de estos conceptos.

\section{Planteamiento del Problema}
Las series de Fourier, desde su concepción por Jean-Baptiste Joseph Fourier a principios del siglo XIX ~\cite{almira2017fourier}, han desempeñado un rol crucial en el análisis y la comprensión de señales y fenómenos periódicos. Las series de Fourier son parte esencial en campos como la ingeniería eléctrica, la física teórica y el procesamiento de señales, imágenes y audio, permitiendo descomponer funciones periódicas en sumas de senos y cosenos, lo que facilita su análisis y manipulación. Estas herramientas matemáticas no solo han impulsado avances significativos en la ciencia y tecnología, transformando nuestra interacción con el mundo, sino que también son cruciales en la enseñanza de los fundamentos teóricos de la ingeniería. Esta capacidad de simplificar señales complejas en componentes básicos no solo mejora el análisis, sino que también facilita la síntesis de nuevas tecnologías que se adaptan a necesidades y entornos cambiantes.

Este amplio espectro de aplicaciones destaca la importancia crítica de las series de Fourier no solo en la investigación avanzada, sino también en la formación académica de futuros ingenieros y científicos. Sin embargo, a pesar de su prevalencia en el currículo educativo, la implementación práctica de este análisis en entornos de aprendizaje a menudo revela áreas de oportunidad dentro de las herramientas disponibles, lo que impacta directamente en la eficacia con la que los estudiantes pueden aplicar y profundizar su comprensión de estos conceptos esenciales. La necesidad de una herramienta más eficiente, como se mencionó en la motivación, es crucial para superar estos desafíos y mejorar la experiencia educativa y profesional en el análisis de series de Fourier.

\section{Justificación}
El análisis de Fourier es fundamental en áreas como la ciencia, la física y la ingeniería, a pesar de esto, el acceso a herramientas que combinen eficientemente el cálculo y la visualización de series de Fourier sigue siendo limitado. Actualmente, hay herramientas que independientemente ayudan a resolver y comprobar algunos problemas pero, ciertos problemas resaltan el valor de esta herramienta ya que los usuarios deben recurrir a múltiples plataformas, lo que añade complejidad y tiempo al proceso, especialmente para aquellos que necesitan resultados rápidos y visualizaciones claras.

Por ello, este proyecto propone el desarrollo de una aplicación web que integre el cálculo simbólico y la representación gráfica interactiva en una sola plataforma. Esto permitirá a estudiantes, ingenieros y profesionales realizar cálculos y visualizar resultados de manera práctica, rápida e intuitiva, sin requerir conocimientos técnicos avanzados o el uso de herramientas separadas. 

\subsection{Encuesta}
Para validar aún mas la relevancia del proyecto, se realizó una encuesta dirigida a estudiantes de ingeniería de diferentes planteles universitarios (véase \hyperref[app1:Encuesta]{Apéndice A}). Los resultados permitieron identificar los principales retos y necesidades al trabajar con asignaturas de matemáticas avanzadas, particularmente con el análisis de Fourier:

A continuación, se presentan los resultados más destacados de la encuesta, organizados por categorías clave. Cada punto incluye el porcentaje de participantes que seleccionaron dicha opción, proporcionando una visión clara de las preferencias y necesidades de los encuestados.
\begin{itemize}
	\item \textbf{Dificultades en asignaturas de matemáticas:} 
	Los encuestados señalaron como principales desafíos:
	\begin{itemize}
		\item Tener la certeza de que sus resultados son correctos (52.4\%).
		\item Entender que es lo que están haciendo (51.2\%).
		\item Resolver muchos ejercicios en periodos cortos de tiempo (46.3\%).
		\item Comprender los conceptos abstractos (46.3\%).
	\end{itemize}
	
	\item \textbf{Uso de software de cálculo y graficación:} 
	La mayoría de los participantes ha utilizado herramientas como \textit{Geogebra} (72\%) y \textit{Symbolab} (53.7\%) las cuales están diseñadas específicamente para ser didácticas y accesibles en el aprendizaje y enseñanza de matemáticas, proporcionando interfaces intuitivas y funcionalidades orientadas a la educación, mientras que herramientas como \textit{MATLAB} (5\%) aunque potentes, están más enfocadas a aplicaciones técnicas avanzadas y tienen una curva de aprendizaje más pronunciada, por lo tanto, son menos populares entre los estudiantes.
	
	\item \textbf{Características esenciales en herramientas digitales:} 
	Los encuestados destacaron la importancia de:
	\begin{itemize}
		\item Resoluciones paso a paso de los problemas (86.6\%).
		\item Interfaces fáciles e intuitivas (69.5\%).
		\item Visualización gráfica interactiva (62.2\%).
	\end{itemize}
	
	\item \textbf{Perspectiva sobre las series de Fourier:}
	\begin{itemize}
		\item El 60\% de los encuestados indicó haber trabajado con problemas relacionados con análisis de Fourier. Dentro de este grupo, los principales retos identificados fueron:
		\begin{itemize}
			\item Dificultad para interpretar sus resultados, una vez llegaban al resultado no sabían que significaba o que hacer con él (56\%).
			\item Los problemas a resolver se tornaban demasiado largos (48\%).
			\item No tuvieron acceso en su momento a herramientas para la resolución de este tipo de problemas (44\%).
			\item No entendieron el proposito de las series de Fourier (40\%).
		\end{itemize}
		\item El 40\% de los encuestados no ha trabajado con Fourier, pero al ver una demostración visual, destacaron que desearían una herramienta que:
		\begin{itemize}
			\item Explique paso a paso los procesos (78.1\%).
			\item Permita ``jugar'' y manipular gráficas interactivas (68.8\%).
			\item Sea fácil de usar (53.1\%).
			\item Sea accesible incluso para quienes no conocen Fourier (46.9\%).
		\end{itemize}
	\end{itemize}
	
	\item \textbf{Importancia de la visualización gráfica:} Ante la pregunta \textit{“¿En qué medida crees que la visualización gráfica interactiva te ayuda a comprender los conceptos?”}, el 56\% de los encuestados calificó con un 10 esta característica, mientras que el 34\% otorgó una puntuación de entre 8 y 9.
\end{itemize}

Los resultados de esta encuesta sugieren una necesidad percibida por los estudiantes de contar con una herramienta que no solo resuelva problemas matemáticos, sino que también integre gráficos interactivos así como la interpretación de los resultados podría facilitar el aprendizaje y la comprensión de las series de Fourier.

\section{Solución Propuesta}
La solución que se plantea en este proyecto es el desarrollo de una aplicación web que permita el cálculo y graficación de series de Fourier de manera integrada, eficiente e intuitiva, cubriendo la brecha existente en las herramientas actuales, que tienden a separar el cálculo matemático de la visualización gráfica en una sola interfaz. La solución propuesta unificará el calculo de los resultados con la graficación interactiva de las mismas en una sola plataforma, diseñada para ser simple de usar, práctica y accesible desde cualquier navegador web, para usuarios que deseen analizar y visualizar series de Fourier, eliminando el problema de usar múltiples programas para resolver y graficar las series de Fourier. \\\\
La aplicación web contará con las siguientes características clave:
\begin{itemize}
	\item \textbf{Cálculo automático de series de Fourier para diferentes tipos de funciones:}
	\vspace{-10pt}
	\begin{itemize}
		\item La aplicación calculará \textbf{series de Fourier} para \textbf{funciones continuas} en un intervalo o para \textbf{funciones definidas a trozos}. El usuario podrá ingresar funciones matemáticas, incluso aquellas con discontinuidades o que estén definidas por partes en distintos intervalos, y la aplicación manejará estos casos automáticamente.
		\item Se implementará tanto la \textbf{serie trigonométrica} como la \textbf{serie exponencial compleja}. La serie trigonométrica descompondrá la función en términos de senos y cosenos, mientras que la versión compleja lo hará en términos de exponenciales imaginarios, lo cual es útil en el contexto de análisis más avanzado y en aplicaciones de ingeniería.
	\end{itemize}
	
	\item \textbf{Extensiones de medio rango:}
	\vspace{-10pt}
	\begin{itemize}
		\item La herramienta permitirá calcular \textbf{extensiones de medio rango} para funciones en un medio intervalo. Esto incluirá la posibilidad de obtener:
		\begin{itemize}
			\item \textbf{Serie de senos} para funciones impares.
			\item \textbf{Serie de cosenos} para funciones pares.
			\item \textbf{Serie completa} para funciones periódicas en un intervalo definido, permitiendo extender la función para construir series de Fourier en un medio rango específico.
		\end{itemize}
	\end{itemize}
	
	\item \textbf{Visualización gráfica interactiva:}
	\vspace{-10pt}
	\begin{itemize}
		\item Una vez calculados los coeficientes de Fourier, la aplicación generará una \textbf{gráfica interactiva} que mostrará la aproximación de la serie de Fourier a la función original. El usuario podrá ajustar el número de términos de la serie para observar cómo mejora la aproximación conforme se incluyen más términos en la suma.
	\end{itemize}
	
	\item \textbf{Interfaz intuitiva y amigable:}
	\vspace{-10pt}
	\begin{itemize}
		\item  La aplicación ofrecerá una interfaz sencilla y accesible, en la que el usuario podrá ingresar las funciones, definir los intervalos y elegir el tipo de serie de Fourier que desea calcular. La experiencia estará diseñada para minimizar la curva de aprendizaje, permitiendo a usuarios sin experiencia en programación obtener resultados rápidamente.
	\end{itemize}
\end{itemize}

\section{Objetivos}
\subsection{Objetivo General}
Desarrollar un prototipo de una aplicación web capaz de calcular las series de Fourier en su forma trigonométrica o exponencial compleja de una función continua en un intervalo o definida a trozos, así como ser capaz de obtener la extensión par o impar de dicha función y, finalmente, graficar tanto la función original como la función expandida como una serie de Fourier y poder añadir o eliminar términos de dicha forma para apreciar como esta se aproxima a la función original.

\subsection{Objetivos Específicos}
\begin{itemize}
	\item Implementar una interfaz de usuario que permita ingresar la función continua o a trozos, seleccionar intervalos y definir el tipo de serie de Fourier (trigonométrica o exponencial) o tipo de expansión (par o impar).
	\item Implementar los módulos para el cálculo de coeficientes de la serie trigonométricas y serie exponencial compleja, adaptándose a funciones continuas o definidas a trozos.
	\item  Desarrollar un módulo de visualización que permita graficar tanto la función original como la
	aproximación de la serie de Fourier. Este módulo debería incluir opciones para añadir o eliminar términos y visualizar cómo estas modificaciones afectan la aproximación a la función original
\end{itemize}
