\chapter{Estado del Arte}\label{ch:Estado del Arte}
En este capitulo...

\section{Estado del Arte}
\begin{table}[h]
	\centering
	\begin{tabular}{ | m{2.5cm} | m{6.5cm} | m{4cm} | }
		\rowcolor{black!75}
		\head {SOFTWARE} & \head {CARACTERÍSTICAS} & \head {PRECIO} \\ \hline
		Wolfram Alpha & Potente motor de conocimiento computacional para cálculos matemáticos y gráficos. No especializado en series de Fourier.~\cite{wolfram2024}  & Desde MXN \$1,200.00 anuales para estudiantes, plan gratuito limitado. \\ \hline
		Geogebra & Herramienta dinámica para construcciones geométricas y gráficas. Requiere cálculos previos para series de Fourier.~\cite{GeoGebra2024} & Software libre y código abierto \\ \hline
		Desmos  & Similar a Geogebra es una calculadora gráfica en línea para cálculos y gráficos, incluida la representación de series de Fourier. Requiere cálculos previos para series de Fourier.~\cite{Desmos2024} & Software libre y código abierto\\ \hline
		Manim & Librería de animación en Python para visualizaciones matemáticas, incluida la animación de series de Fourier. Requiere conocimientos de programación y cálculos previos.~\cite{Manim2024} & Software libre y código abierto \\ \hline
		Matlab  & Entorno de programación para cálculos numéricos y visualización de datos, con herramientas específicas para series de Fourier. Requiere conocimientos en programación.~\cite{MathWorks2024} & Desde USD\$99 (aprox. MXN\$1627.82) anuales para estudiantes.\\ \hline
	\end{tabular}
	\caption{Comparación de software para cálculos matemáticos y visualización de datos}
	\label{tabla:software}
\end{table}
