\chapter{Marco Teórico}\label{ch:Marco Teórico}
Las series de Fourier tienen sus raíces en el estudio de la propagación del calor y el desarrollo de la teoría analítica del calor a principios del siglo XIX. Este campo de estudio fue fundamental para el avance de las matemáticas aplicadas, particularmente en el contexto de las ecuaciones diferenciales que modelan fenómenos físicos.

\section{Origen e Historia}
Jean-Baptiste Joseph Fourier (1768-1830), matemático y físico francés, es considerado el pionero en el análisis de la conducción de calor. En su obra seminal, "Théorie Analytique de la Chaleur" (Teoría Analítica del Calor) publicada en 1822, Fourier abordó el problema de cómo el calor se distribuye y se propaga en un medio sólido a lo largo del tiempo.


\section{Recommendations and future work}



\begin{table}[hbtp]
	\centering
	\begin{tabular}{@{}*{2}{p{0.5\textwidth}}@{}}
		\toprule
		\textbf{Correct} &  \textbf{Incorrect}
		\\
		\midrule
		\enquote{This is an \enquote{inner quote} inside an outer quote}
		&
		"This is an 'inner quote' inside an outer quote"
		\\
		\bottomrule
	\end{tabular}
	\caption[Quotation marks]
	{Proper quotation mark usage.
		The \texttt{\textbackslash enquote} command chooses the correct
		quotation marks for the specified language.}
\end{table}
\lipsum[1]