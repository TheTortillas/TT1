\chapter{Trabajo a futuro}\label{ch:Trabajo-a-futuro}
Las actividades planeadas para este proyecto durante el plazo del siguiente Trabajo Terminal II son las siguientes:

\begin{itemize}
		\item \textbf{Completar la API}: Ya teniendo los primeros endpoints de la API, se desarrollarán los faltantes para calcular las extensiones de medio rango, así como procesar funciones a trozos. Además de tratar con excepciones.
		
		\item \textbf{Implementar forma Amplitud - Fase}: La API contará con un nuevo endpoint para calcular la serie de Fourier en su forma de amplitud - fase, así como crear las funciones para graficar este tipo de funciones en el lienzo.
		
		\item \textbf{Diseñar interfaz para el lienzo}: Se creará una interfaz para colocar el lienzo, además de secciones para mostrar los resultados y los controles para calcular la serie de Fourier.
		
		\item \textbf{Crear un catalogo de funciones comunes}: A lo largo del estudio de las series de Fourier se presentan problemas comunes como la función diente de sierra, la función escalón, la función piso, un tren de pulsos, estas funciones debe estar en un catalogo para calcularse inmediatamente.
		
		\item \textbf{Implementar un parser Maxima a Javascript}: Se diseñará e implementara un parser para convertir las cadenas de formato Maxima a formato Javascript, para que el lienzo pueda interpretar y graficar los resultados devueltos por la API.
		
		\item \textbf{Unificar todos los módulos}: Ya teniendo los módulos de graficación, API de calculo simbólico, parseo de datos, e interfaz de ingreso de datos, todas se unificarán en un solo proyecto de Angular para que operen entre si.
		
		\item \textbf{Despliegue y pruebas}: Ya teniendo la apliación funcionando, se desplegará sobre la plataforma de servicios de nube de Microsoft Azure, además de obtener un dominio temporal y un certificado SSL para ser compartida con alumnos y profesores para probar su funcionalidad y rendimiento.
\end{itemize}