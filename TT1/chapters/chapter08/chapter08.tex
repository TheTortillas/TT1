\chapter{Conclusiones}\label{ch:Trabajo-a-futuro}
El desarrollo de la aplicación web cumplió satisfactoriamente con el objetivo general planteado en este proyecto: crear una herramienta capaz de calcular y graficar series de Fourier. A partir de la identificación de un área de oportunidad en el análisis y visualización de estas series, se diseñó e implementó una solución funcional que permite obtener los coeficientes en sus distintas formas —trigonométrica periódica, expansiones de medio rango en senos o cosenos, y forma exponencial compleja—. Los resultados numéricos se presentan en formato LaTeX. A partir de estos cálculos, la aplicación facilita la visualización gráfica tanto de la función original como de su aproximación mediante series de Fourier, permitiendo también representar individualmente cada armónico.

La metodología de desarrollo mediante prototipos permitió validar desde etapas tempranas la viabilidad técnica del proyecto. En la primera iteración, las pruebas individuales de tecnologías así como las pequeñas integraciones sirvieron para establecer una base sólida; en la segunda, se concretó el desarrollo completo de la aplicación y su despliegue en la nube, garantizando accesibilidad desde cualquier navegador. La interfaz gráfica se diseñó con un enfoque en la usabilidad, permitiendo que tanto estudiantes como docentes pudieran utilizar la herramienta sin requerir conocimientos avanzados en programación, lo cual fue confirmado durante las pruebas realizadas por usuarios reales.

Un aspecto destacable del proyecto fue la superación de los objetivos inicialmente definidos. No solo se logró una correcta implementación de las funcionalidades previstas, sino que además se abordaron casos no contemplados en el planteamiento original. En particular, se identificaron funciones cuya manipulación simbólica dentro del entorno de desarrollo presentaba dificultades para calcular sus coeficientes mediante integrales analíticas tradicionales. Ante estas limitaciones del software, se incorporó una solución alternativa basada en la Transformada Discreta de Fourier (DFT) y su inversa (IDFT), lo que amplió significativamente la capacidad de análisis de la herramienta y fortaleció su aplicabilidad práctica.



