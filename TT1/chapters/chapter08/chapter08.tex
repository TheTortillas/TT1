\chapter{Conclusiones parciales}\label{ch:conclusiones}
El proyecto ``Desarrollo de una apliación web para el calculo y graficación de series de Fourier'' ha demostrado tener potencial para alumnos de áreas de ingeniería dentro de asignaturas que involucren problemas sobre análisis de Fourier. Si bien, aún falta trabajo para completar este proyecto, se han sentado las bases para confirmar que el desarrollo puede llevarse acabo sin problemas de compatibilidad entre sus diferentes módulos, que era mi mayor temor.
\newline
Al ser un equipo de un solo integrante la administración de los tiempos para el desarrollo, así como carga de otras actividades o factores externos son el mayor enemigo para completar este proyecto y han sido un gran reto, además de la integración de tecnologías como Angular, NodeJS y Maxima requirió esfuerzo extra al ser tecnologías con las que no estaba tan familiarizado, para lo que se dedico tiempo, sin embargo, es un tema que yo propuse y que me apasiona, lo que ha servido como una fuente constante de motivación para superar estos desafíos. La pasión por el tema ha mantenido la constancia necesaria para avanzar en el desarrollo, a pesar de las limitaciones de tiempo y personal.
\newline
Este sistema planea ofrecer un alivio a los alumnos de ingeniería de, no solo nuestra escuela, si no de cualquiera que requiera calcular de una serie de Fourier para una tarea o para algún necesidad laboral, o si simplemente quiere entrar para introducirse o jugar con las series de Fourier, le sea de utilidad.
