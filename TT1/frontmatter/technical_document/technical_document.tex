\chapter{Resumen}
El presente trabajo terminal desarrolla una aplicación web para el cálculo y visualización interactiva de series de Fourier, ofreciendo una solución integrada que combina el procesamiento matemático y la representación gráfica en una sola plataforma. La aplicación permite trabajar con series de Fourier tanto en su forma trigonométrica como exponencial compleja, procesando funciones continuas y definidas a trozos, así como sus extensiones pares e impares. Adicionalmente, implementa el cálculo de la Transformada Discreta de Fourier (DFT) como alternativa para aquellas funciones que no pueden integrarse analíticamente mediante las fórmulas tradicionales de la serie de Fourier.

La arquitectura del sistema se basa en un modelo cliente-servidor, donde el frontend desarrollado en Angular/TypeScript proporciona una interfaz intuitiva para la entrada de funciones y la visualización de resultados, mientras que el backend implementado en Node.js se comunica con Maxima para realizar los cálculos simbólicos necesarios. Esta estructura modular facilita tanto el mantenimiento como la escalabilidad del sistema.

La interfaz incluye un componente gráfico interactivo que permite a los usuarios visualizar en tiempo real cómo la serie de Fourier aproxima la función original a medida que se aumenta el número de términos, proporcionando una herramienta valiosa para la comprensión conceptual del análisis de Fourier. Según los estudios realizados mediante encuestas a estudiantes, esta aproximación integrada aborda directamente las dificultades que enfrentan al trabajar con este tipo de análisis matemático.

Esta herramienta ha sido diseñada como apoyo educativo para estudiantes y docentes en asignaturas de matemáticas avanzadas e ingeniería, facilitando tanto la resolución de problemas como la comprensión visual de los conceptos fundamentales del análisis de Fourier.

%Los módulos principales que contiene la aplicación son:

%\begin{enumerate}
%	\item \textbf{Módulo para introducir la función:} ya sea definida en una sola función o a trozos, además de su periodo. Además, se tienen opciones para ingresar extensiones de funciones de medio rango pares, impares.
%	\item \textbf{Módulo para el cálculo de coeficientes:} donde se realizarán los cálculos correspondientes a los coeficientes mediante integración simbólica.
%	\item \textbf{Módulo de graficación de funciones periódicas y coeficientes de la serie:} donde se podrá ver la aproximación de la serie a la función mientras se incrementa el número de términos de la expansión.
%\end{enumerate}

\vspace{0.5cm}

\textbf{Palabras Clave -} aplicación web, análisis de Fourier, cálculo, matemáticas avanzadas para la ingeniería.

\vspace{0.5cm}

\textbf{Correo de Contacto} \\
smoralesp1700@alumno.ipn.mx \\