\chapter{Documento Técnico}
\begin{center}
	\textbf{\Large “Desarrollo de una aplicación web para el cálculo y graficación de series de Fourier”}
	\vspace{0.5cm}
\end{center}

\textbf{Resumen -} Dentro del presente trabajo se lleva a cabo el desarrollo del prototipo de una aplicación web diseñada para calcular y graficar el desarrollo en series de Fourier de funciones periódicas. El desarrollo en serie puede ser tanto en forma trigonométrica como en forma exponencial compleja. Los módulos principales que contiene la aplicación son:

\begin{enumerate}
	\item \textbf{Módulo para introducir la función} ya sea definida en una sola función o a trozos, junto de su periodo. Además, se tienen opciones para ingresar extensiones de funciones de medio rango pares, impares.
	\item \textbf{Módulo para el cálculo de coeficientes} donde se realizarán los cálculos correspondientes a los coeficientes mediante integración simbólica usando Maxima
	\item \textbf{Módulo de graficación de funciones periódicas y coeficientes de la serie}, donde se podrá ver la aproximación de la serie a la función mientras se incrementa el número de términos de la expansión.
\end{enumerate}

 La aplicación pretende ser de utilidad como herramienta de apoyo para estudiantes y docentes en el tema del análisis de Fourier.

\vspace{0.5cm}

\textbf{Palabras Clave -} aplicación web, análisis de Fourier, cálculo, matemáticas avanzadas para la ingeniería.

\vspace{0.5cm}

\textbf{Correo de Contacto -} \\
smoralesp1700@alumno.ipn.mx \\