\chapter{Documento Técnico}
\begin{center}
	\textbf{\Large “Desarrollo de una aplicación web para el cálculo y graficación de series de Fourier”}
	\vspace{0.5cm}
\end{center}

\textbf{Resumen -} Se propone desarrollar una aplicación web para calcular y graficar el desarrollo en series de Fourier de funciones periódicas, las funciones pueden ser discretas o continuas. El desarrollo en serie puede ser tanto en forma trigonométricas como en forma exponencial compleja. Los módulos principales que tendrá la aplicación son: 1. Modulo para introducir la función de diferentes maneras, 2. Módulo para el cálculo de coeficientes mediante integración analítica o por métodos numéricos, 3. Módulo de graficación de funciones periódicas y coeficientes de la serie, donde se podrá ver la aproximación de la serie a la función mientras se incrementa el número de términos de la expansión. Se ofrecerán opciones para calcular extensiones de funciones pares e impares. La aplicación puede ser de utilidad como herramienta de apoyo para estudiantes y docentes en
el tema de análisis de Fourier.

\vspace{0.5cm}

\textbf{Palabras Clave -} aplicación web, análisis de Fourier, cálculo, matemáticas avanzadas para la ingeniería.

\vspace{0.5cm}

\textbf{Correo de Contacto -} \\
smoralesp1700@alumno.ipn.mx \\