\chapter{Resumen}
El presente trabajo terminal desarrolla una aplicación web para el cálculo y visualización interactiva de series de Fourier, ofreciendo una solución integrada que combina el procesamiento matemático y la representación gráfica en una sola plataforma. La aplicación permite trabajar con series de Fourier tanto en su forma trigonométrica como exponencial compleja, procesando funciones continuas y definidas a trozos, así como sus extensiones pares e impares. Adicionalmente, implementa el cálculo de la Transformada Discreta de Fourier (DFT) como alternativa para aquellas funciones que no pueden integrarse analíticamente mediante las fórmulas tradicionales de la serie de Fourier.

La arquitectura del sistema se basa en un modelo cliente-servidor, donde el frontend desarrollado en Angular/TypeScript proporciona una interfaz intuitiva para la entrada de funciones y la visualización de resultados, mientras que el backend implementado en Node.js se comunica con Maxima para realizar los cálculos simbólicos necesarios. Esta estructura modular facilita tanto el mantenimiento como la escalabilidad del sistema.

La interfaz incluye un componente gráfico interactivo que permite a los usuarios visualizar en tiempo real cómo la serie de Fourier aproxima la función original a medida que se aumenta el número de términos, proporcionando una herramienta valiosa para la comprensión conceptual del análisis de Fourier. Según los estudios realizados mediante encuestas a estudiantes, esta aproximación integrada aborda directamente las dificultades que enfrentan al trabajar con este tipo de análisis matemático.

Esta herramienta ha sido diseñada como apoyo educativo para estudiantes y docentes en asignaturas de matemáticas avanzadas e ingeniería, facilitando tanto la resolución de problemas como la comprensión visual de los conceptos fundamentales del análisis de Fourier.

\vspace{0.5cm}

\textbf{Palabras Clave -} aplicación web, análisis de Fourier, cálculo, matemáticas avanzadas para la ingeniería.

\vspace{0.5cm}

\textbf{Correo de Contacto} \\
smoralesp1700@alumno.ipn.mx \\

\cleardoublepage

\chapter{Abstract}
This terminal project presents a web application for the calculation and interactive visualization of Fourier series, offering an integrated solution that combines mathematical processing and graphical representation within a single platform. The application supports both the trigonometric and complex exponential forms of Fourier series, processing continuous and piecewise-defined functions, as well as their even and odd extensions. Additionally, it implements the Discrete Fourier Transform (DFT) as an alternative for functions that cannot be analytically integrated using traditional Fourier series formulas.

The system architecture is based on a client-server model, where the frontend developed in Angular/TypeScript provides an intuitive interface for function input and result visualization, while the backend implemented in Node.js communicates with Maxima to perform the required symbolic computations. This modular structure facilitates both maintenance and scalability of the system.

The interface includes an interactive graphical component that allows users to visualize in real-time how the Fourier series approximates the original function as the number of terms increases, offering a valuable tool for conceptual understanding of Fourier analysis. According to surveys conducted among students, this integrated approach directly addresses the challenges they face when working with this type of mathematical analysis.

This tool has been designed as an educational aid for students and instructors in advanced mathematics and engineering courses, facilitating both problem-solving and the visual comprehension of the fundamental concepts of Fourier analysis.

\vspace{0.5cm}

\textbf{Keywords -} web application, Fourier analysis, computation, advanced mathematics for engineering..

\vspace{0.5cm}

\textbf{Contact email} \\
smoralesp1700@alumno.ipn.mx \\